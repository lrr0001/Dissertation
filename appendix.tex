\begin{appendices}

%Some Table of Contents entry formatting
\addtocontents{toc}{\protect\renewcommand{\protect\cftchappresnum}{\appendixname\space}}
\addtocontents{toc}{\protect\renewcommand{\protect\cftchapnumwidth}{6em}}

%Begin individual appendices, separated as chapters

\chapter{Hermitian Rank-1 Updates for the Cholesky Decomposition} \label{chapter:Cholesky}
These derivations are based on the work of \cite{krause2015more}, but modified to handle complex numbers.


Let $\mA^{(n)} = \mL\mL^H$ be a Hermitian, positive definite matrix of dimension $C \times C$ and $\mL$ be a lower trangular matrix. Then, the diagonal of $\mL$ is positive and real.

\begin{equation}
\mA^{(n)} = \begin{bmatrix} a_{1,1} & \hdots & a_{C,1}^{*} \\
                      \vdots & \ddots & \vdots \\
                      a_{C,1} & \hdots & a_{C,C}
      \end{bmatrix}
\end{equation}

\begin{equation}
\mL = \begin{bmatrix} \ell_{1,1} & 0      & \hdots & 0 \\
                      \ell_{2,1} & \ddots &        & \vdots \\
                      \vdots     &        &  \ddots &  0     \\
                      \ell_{C,1}     & \hdots &   & \ell_{C,C}
      \end{bmatrix}
\end{equation}

Dividing $\mA^{(n)}$ into blocks:
\begin{equation}
\mA^{(n)} = \begin{bmatrix}\ell_{1,1} & \vzero^T \\ \vl_{2,1} & \mL_{2,2} \end{bmatrix} \begin{bmatrix} \ell_{1,1}^{*} & \vl_{2,1}^H \\ \vzero & \mL_{2,2}^H \end{bmatrix} 
\end{equation}

\begin{equation}
\mA^{(n)} = \begin{bmatrix} \ell_{1,1}^2 & \ell_{1,1}\vl_{2,1}^H \\ \ell_{1,1}\vl_{2,1} & \mL_{2,2}\mL_{2,2}^H + \vl_{2,1}\vl_{2,1}^H \end{bmatrix}
\end{equation}

Consider the rank-$1$ update:\footnote{If $\lambda$ is negative, the result is not guarenteed to be positive definite. In general, updates to inverse representations with a negative $\lambda$ (sometimes refered to as "downdates") are less numerically stable, even if the resulting matrix is still positive definite.}
\begin{equation}
\mA^{(n + 1)} = \mA^{(n)} + \lambda\vv\vv^H
\end{equation}

where
\begin{equation}
\vv = \begin{bmatrix} v_1 \\ \vv_2 \end{bmatrix}
\end{equation}

\begin{equation}
\mA^{(n + 1)} = \begin{bmatrix} \ell_{1,1}^2 + \lambda v_1 v_1^{*} & \ell_{1,1}\vl_{2,1}^H + \lambda v_1\vv_2^H \\ \ell_{1,1}\vl_{2,1} + \lambda v_1^{*}\vv_2 & \mL_{2,2}\mL_{2,2}^H + \vl_{2,1}\vl_{2,1}^H + \lambda\vv_2\vv_2^H \end{bmatrix}
\end{equation}

Let $\mA^{(n + 1)} = \mM\mM^H$, where $\mM$ is a lower triangular matrix. Then, the diagonal of $\mM$ is positive and real.

\begin{equation}
\mA^{(n + 1)} = \begin{bmatrix} m_{1,1}^2 & m_{1,1}\vm_{2,1}^H \\ m_{1,1}\vm_{2,1} & \mM_{2,2}\mM_{2,2}^H + \vm_{2,1}\vm_{2,1}^H \end{bmatrix}
\end{equation}

Therefore,
\begin{equation}
m_{1,1}^2 = \ell_{1,1}^2 + \lambda v_1 v_1^{*}
\end{equation}
\begin{equation}
m_{1,1}\vm_{2,1} = \ell_{1,1}\vl_{2,1} + \lambda v_1^{*}\vv_2
\end{equation}
\begin{equation}
\mM_{2,2}\mM_{2,2}^H + \vm_{2,1}\vm_{2,1}^H = \mL_{2,2}\mL_{2,2}^H + \vl_{2,1}\vl_{2,1}^H + \lambda\vv_2\vv_2^H
\end{equation}

Solving for the first column of $\mM$:
\begin{equation} \label{equation:m11}
m_{1,1} = \sqrt{\ell_{1,1}^2 + \lambda v_1 v_1^{*}}
\end{equation}
\begin{equation}
\vm_{2,1} = \frac{\ell_{1,1}\vl_{2,1} + \lambda v_1^{*}\vv_2}{m_{1,1}}
\end{equation}
Finally,
\begin{equation}
\mM_{2,2}\mM_{2,2}^H = \mL_{2,2}\mL_{2,2}^H + \vl_{2,1}\vl_{2,1}^H + \lambda\vv_2\vv_2^H - \vm_{2,1}\vm_{2,1}^H
\end{equation}

\begin{equation}
\vm_{2,1}\vm_{2,1}^H = \frac{1}{m_{1,1}^2}\left(\ell_{1,1}^2\vl_{2,1}\vl_{2,1}^H + \lambda\ell_{1,1}v_1\vl_{2,1}\vv_2^H + \lambda\ell_{1,1}v_1^{*}\vv_2\vl_{2,1}^H + \lambda^2 v_1 v_1^{*} \vv_2\vv_2^H\right)
\end{equation}

\begin{equation}
\mM_{2,2}\mM_{2,2}^H = \mL_{2,2}\mL_{2,2}^H + \frac{m_{1,1}^2 - \ell_{1,1}^2}{m_{1,1}^2}\vl_{2,1}\vl_{2,1}^H + \frac{\lambda(m_{1,1}^2 - \lambda v_1 v_1^{*})}{m_{1,1}^2}\vv_2\vv_2^H - \frac{\lambda\ell_{1,1}v_1}{m_{1,1}^2}\vl_{2,1}\vv_2^H - \frac{\lambda\ell_{1,1}v_1^{*}}{m_{1,1}}\vv_2\vl_{2,1}^H
\end{equation}

The expressions $m_{1,1}^2 - \lambda v_1 v_1^{*}$ and $m_{1,1}^2 - \ell_{1,1}^2$ can be simplified using equation \ref{equation:m11}.
\begin{equation}
\mM_{2,2}\mM_{2,2}^H = \mL_{2,2}\mL_{2,2}^H + \frac{\lambda v_1 v_1^{*} }{m_{1,1}^2}\vl_{2,1}\vl_{2,1}^H + \frac{\lambda \ell_{1,1}^2}{m_{1,1}^2}\vv_2\vv_2^H - \frac{\lambda\ell_{1,1}v_1}{m_{1,1}}\vl_{2,1}\vv_2^H - \frac{\lambda\ell_{1,1}v_1^{*}}{m_{1,1}^2}\vv_2\vl_{2,1}^H
\end{equation}

Factoring out $\frac{\lambda}{m_{1,1}^2}$:
\begin{equation}
\mM_{2,2}\mM_{2,2}^H = \mL_{2,2}\mL_{2,2}^H + \frac{\lambda}{m_{1,1}^2}\left( v_1 v_1^{*} \vl_{2,1}\vl_{2,1}^H + \ell_{1,1}^2\vv_2\vv_2^H - \ell_{1,1}v_1\vl_{2,1}\vv_2^H - \ell_{1,1}v_1^{*}\vv_2\vl_{2,1}^H\right)
\end{equation}

Note the factorization:
\begin{equation}
(\ell_{1,1}\vv_2 - v_1\vl_{2,1})(\ell_{1,1}\vv_2 - v_1\vl_{2,1})^H = \ell_{1,1}^2\vv_2\vv_2^H - \ell_{1,1}v_1^{*}\vl_{2,1}^H - \ell_{1,1}v_1\vl_{2,1}\vv_2 + v_1 v_1^{*}\vl_{2,1}\vl_{2,1}^H
\end{equation}
Therefore,
\begin{equation}
\mM_{2,2}\mM_{2,2}^H = \mL_{2,2}\mL_{2,2}^H + \frac{\lambda}{m_{1,1}^2}\left(\ell_{1,1}\vv_2 - v_1\vl_{2,1}\right)\left(\ell_{1,1}\vv_2 - v_1\vl_{2,1}\right)^H
\end{equation}
$\mL_{2,2}\mL_{2,2}^H$ is a $ (C - 1) \times (C - 1)$ Hermitian, positive definite matrix and $\frac{\lambda}{m_{1,1}^2}(\ell_{1,1}\vv_2 - v_1\vl_{2,1})(\ell_{1,1}\vv_2 - v_1\vl_{2,1})^H$ is a rank-$1$ Hermitian update, so the process can be repeated on subsequent columns of $\mL$ until the entire Cholesky decomposition has been updated. Each column update is computed in linear time, so the entire update can be computed in quadratic time.

While the order of complexity cannot be further reduced, there are changes that can be made to decrease precision error. Factoring out $\ell_{1,1}$:
\begin{equation}
\mM_{2,2}\mM_{2,2}^H = \mL_{2,2}\mL_{2,2}^H + \frac{\lambda\ell_{1,1}^2}{m_{1,1}^2}\left(\vv_2 - \frac{v_1}{\ell_{1,1}}\vl_{2,1}\right)\left(\vv_2 - \frac{v_1}{\ell_{1,1}}\vl_{2,1}\right)^H
\end{equation}

Rather than directly updating $\lambda$ before moving on to the next column, better precision can be achieved by updating a divisor instead.  Looking at the fraction $\frac{\ell_{1,1}^2}{m_{1,1}^2}$:
\begin{equation}
\frac{\ell_{1,1}^2}{m_{1,1}^2} = \frac{\ell_{1,1}^2}{\ell_{1,1}^2 + \lambda v_1 v_1^{*}}
\end{equation}

\begin{equation}
\frac{\ell_{1,1}^2}{m_{1,1}^2} = \frac{1}{1 + \frac{\lambda v_1 v_1^{*}}{\ell_{1,1}^2}}
\end{equation}

Therefore,
\begin{equation}
\mM_{2,2}\mM_{2,2}^H = \mL_{2,2}\mL_{2,2}^H + \frac{\lambda}{1 + \frac{\lambda v_1 v_1^{*}}{\ell_{1,1}^2}}(\vv_2 - \frac{v_1}{\ell_{1,1}}\vl_{2,1})(\vv_2 - \frac{v_1}{\ell_{1,1}}\vl_{2,1})^H
\end{equation}

Let
\begin{equation}
\vomega = \vv_2 - \frac{v_1}{\ell_{1,1}}\vl_{2,1}
\end{equation}

So,
\begin{equation}
\mM_{2,2}\mM_{2,2}^H = \mL_{2,2}\mL_{2,2}^H + \frac{\lambda}{1 + \frac{\lambda v_1 v_1^{*}}{\ell_{1,1}^2}}\vomega\vomega^H
\end{equation}

Previously, $\vm_{2,1}$ was written in terms of $\vl_{2,1}$ and $\vv$. It can instead be written in terms of $\vl_{2,1}$ and $\vomega$. Recall that
\begin{equation}
\vm_{2,1} = \frac{\ell_{1,1}\vl_{2,1} + \lambda v_1^{*}\vv_2}{m_{1,1}}
\end{equation}

Substituting for $\vv_2$:
\begin{equation}
\vm_{2,1} = \frac{\ell_{1,1}\vl_{2,1} + \lambda v_1^{*}(\vomega + \frac{v_1}{\ell_{1,1}}\vl_{2,1})}{m_{1,1}}
\end{equation}

Combining the $\vl_{2,1}$ terms:
\begin{equation}
\vm_{2,1} = \frac{(\ell_{1,1} +  \frac{\lambda v_1 v_1^{*}}{\ell_{1,1}}) \vl_{2,1} + \lambda v_1^{*}\vomega}{m_{1,1}}
\end{equation}

Factoring out $\frac{1}{\ell_{1,1}}$ produces the expression for $m_{1,1}^2$.
\begin{equation}
\vm_{2,1} = \frac{\frac{1}{\ell_{1,1}}(\ell_{1,1}^2 +  \lambda v_1 v_1^{*}) \vl_{2,1} + \lambda v_1^{*}\vomega}{m_{1,1}}
\end{equation}
\begin{equation}
\vm_{2,1} = \frac{\frac{m_{1,1}^2}{\ell_{1,1}} \vl_{2,1} + \lambda v_1^{*}\vomega}{m_{1,1}}
\end{equation}

\begin{equation}
\vm_{2,1} = \frac{m_{1,1}}{\ell_{1,1}} \vl_{2,1} + \frac{\lambda v_1^{*}}{m_{1,1}}\vomega
\end{equation}

So, to summarize:
\begin{equation} \label{equation:m11}
m_{1,1} = \sqrt{\ell_{1,1}^2 + \lambda |v_1|^2}
\end{equation}

\begin{equation}
\vomega = \vv_2 - \frac{v_1}{\ell_{1,1}}\vl_{2,1}
\end{equation}

\begin{equation}
\vm_{2,1} = \frac{m_{1,1}}{\ell_{1,1}} \vl_{2,1} + \frac{\lambda v_1^{*}}{m_{1,1}}\vomega
\end{equation}

\begin{equation}
\mM_{2,2}\mM_{2,2}^H = \mL_{2,2}\mL_{2,2}^H + \frac{\lambda}{1 + \frac{\lambda |v_1|^2}{\ell_{1,1}^2}}\vomega\vomega^H
\end{equation}

The rank-$1$ Hermitian update to a Cholesky decomposition can be computed in quadratic time. The algorithm can be rewritten to be computed in place.

\begin{algorithm}[H]
\SetAlgoLined
 %initialization:\\
   $\vomega \leftarrow \vv$ \\
   $\alpha \leftarrow 1$ \\
  \For{$i \in \{1,\hdots,C\}$}{
     $\eta \leftarrow \alpha\ell_{i,i}^2 + \lambda |\omega_i|^2$ \\
     $m_{i,i} \leftarrow \sqrt{\ell_{i,i}^2 + \frac{\lambda|\omega_i|^2}{\alpha}}$ \\
     \For{$j \in \{i + 1,\hdots, C\}$}{
        $\omega_j \leftarrow \omega_j - \frac{\ell_{j,i}\omega_i}{\ell_{i,i}} $ \\
        $m_{j,i} \leftarrow \frac{m_{i,i}\ell_{j,i}}{\ell_{i,i}} + \frac{\lambda m_{i,i}\omega_j\omega_i^{*}}{\eta}$
}
     $\alpha \leftarrow \alpha + \frac{\lambda|\omega_i|^2}{\tau^2}$
}
 \caption{Cholesky Decomposition Hermitian Rank-One Update}
\end{algorithm} 




\begin{algorithm}[H]
\SetAlgoLined
 %initialization:\\
   $\vomega \leftarrow \vv$ \\
   $\alpha \leftarrow 1$ \\
  \For{$i \in \{1,\hdots,C\}$}{
     $\eta \leftarrow \alpha\ell_{i,i}^2 + \lambda |\omega_i|^2$ \\
     $\tau \leftarrow \ell_{i,i}$ \\
     $\ell_{i,i} \leftarrow \sqrt{\ell_{i,i}^2 + \frac{\lambda|\omega_i|^2}{\alpha}}$ \\
     \For{$j \in \{i + 1,\hdots, C\}$}{
        $\omega_j \leftarrow \omega_j - \frac{\ell_{j,i}\omega_i}{\tau} $ \\
        $\ell_{j,i} \leftarrow \frac{\ell_{i,i}\ell_{j,i}}{\tau} + \frac{\lambda\ell_{i,i}\omega_j\omega_i^{*}}{\eta}$
}
     $\alpha \leftarrow \alpha + \frac{\lambda|\omega_i|^2}{\tau^2}$
}
 \caption{Cholesky Decomposition Hermitian Rank-One Update (In Place)}
\end{algorithm} 


\chapter{Rank-2 Eigendecomposition Edge Cases} \label{chapter:eigenedge}

\section{Less than 2 Independent Eigenvectors}

\begin{equation}
\mB = \begin{bmatrix} \mD^H\vu & \vv\end{bmatrix}^H
\end{equation}


The matrix $\mA\mB$ is a Hermitian $M \times M$ matrix, so it has $M$ real eigenvalues and $M$ independent eigenvectors which can be chosen to be orthogonal. Therefore, $\mB\mA$ has $2$ independent eigenvectors if $\mB$ is rank $2$.  There are three cases that can cause $\mB$ to have a rank less than $2$.


\begin{enumerate}
\item
\begin{equation}
\mD^H\vu = \vzero
\end{equation}
\item

\begin{equation}
\vv = \vzero
\end{equation}
\item
\begin{equation}
\mD^H\vu = \alpha \vv
\end{equation}
\end{enumerate}

In the first $2$ cases, the matrix $\mB\mA$ has one independent eigenvector. The first case implies that only the Hermitian update is nonzero.  The second case implies that the entire update is zero. (The eigenvalues are zero, so as long as the normalization of eigenvectors is handled with care, it is not necessary to check for these cases in code.)

In the third case, the diagonalization of $\mA\mB$ can be determined directly without using the $2 \times 2$ matrix $\mB\mA$.

\begin{equation}
\mA\mB = 2\operatorname{real}(\alpha)\vv\vv^H
\end{equation}

\begin{equation}
\lambda = 2\operatorname{real}(\alpha) \|\vv\|_2^2
\end{equation}

\begin{equation}
\vx = \frac{\vv}{\|\vv\|_2}
\end{equation}

The rest of the eigenvalues are zero.  (The corresponding $2 \times 2$ matrix $\mB\mA$ shares the same nonzero eigenvalue. The eigenvector that is lost has an eigenvalue of zero, so like in the other $2$ cases, it is not necessary to check for this case in code.)

\section{Eigenvalues are Not Distinct}

If the pair of eigenvalues are the same, then all nonzero vectors are eigenvectors of $\mB\mA$. However, it is necessary for a diagonalization expansion with only Hermitian terms that the eigenvectors of $\mA\mB$ are chosen to be orthogonal, which can be found using a Gram-Schmidt process. This case is not just a theoretical concern; it is necessary to check for this in code.

\end{appendices}
