\documentclass{article}
%\usepackage{amsmath}
%\usepackage{bm}
%\usepackage{bbm}
\usepackage{graphicx}
\usepackage{amsmath}
\usepackage{amsfonts}
\usepackage{bm}
\usepackage{bbm}
\usepackage{mathrsfs}
% fields
\newcommand{\R}{\mathbb{R}}
\newcommand{\C}{\mathbb{C}}
\newcommand{\Z}{\mathbb{Z}}
\newcommand{\N}{\mathbb{N}}

% complex numbers
\renewcommand{\Re}[1]{\operatorname{Re}\left\{#1\right\}}
\renewcommand{\Im}[1]{\operatorname{Im}\left\{#1\right\}}
\newcommand{\conj}[1]{\mkern 1.5mu\overline{\mkern-1.5mu#1\mkern-1.5mu}\mkern 1.5mu}

% probability and stat
\renewcommand{\P}[1]{\operatorname{P}\left(#1\right)}
\newcommand{\E}{\operatorname{E}}
\newcommand{\var}{\operatorname{var}}

% calculus 
\renewcommand{\d}[1]{d#1}

% constants (written in roman, if wanted)
\newcommand{\e}{e}
\renewcommand{\j}{j}

% linear algebra
% 	vector notation
%\newcommand{\vct}[1]{\boldsymbol{#1}}
\newcommand{\vct}[1]{\bm{#1}}
%   matrices
%\newcommand{\mtx}[1]{\boldsymbol{#1}}
\newcommand{\mtx}[1]{\bm{#1}}
% diagonal
\newcommand{\diag}{\operatorname{diag}}
%   block vector
\newcommand{\bvct}[1]{\mathbf{#1}}
%   block matrix
\newcommand{\bmtx}[1]{\mathbf{#1}}
%	inner products
\newcommand{\<}{\langle}
\renewcommand{\>}{\rangle}
% 	transpose, Hermitian, pseudo-inverse
\renewcommand{\H}{\mathrm{H}}
\newcommand{\T}{\mathrm{T}}
\newcommand{\pinv}{\dagger}
%	fundamental subspaces
\newcommand{\Null}{\operatorname{Null}}
\newcommand{\Range}{\operatorname{Range}}
\newcommand{\Span}{\operatorname{Span}}
%	operators
\newcommand{\trace}{\operatorname{trace}}
\newcommand{\rank}{\operatorname{rank}}
\newcommand{\F}{\mathcal{F}}

% sets and topology
\newcommand{\set}[1]{\mathcal{#1}}
\newcommand{\closure}{\operatorname{cl}}	% closure
\newcommand{\interior}{\operatorname{int}}
\newcommand{\boundary}{\operatorname{bd}}
\newcommand{\diameter}{\operatorname{diam}}

% functional analysis
\newcommand{\domain}{\operatorname{dom}}
\newcommand{\epigraph}{\operatorname{epi}}
\newcommand{\hypograph}{\operatorname{hypo}}
\newcommand{\linop}[1]{\mathscr{#1}}	% general linear operator

% optimization
\renewcommand{\L}{\mathcal{L}}
\DeclareMathOperator*{\minimize}{\text{minimize}}
\DeclareMathOperator*{\maximize}{\text{maximize}}
\newcommand{\indicator}{\mathbbm{1}}
\newcommand{\prox}{\operatorname{prox}}

% Complexity
\newcommand{\bigO}{\mathcal{O}}

% Neural Networks
\newcommand{\pool}{\mtx{P}}


%--------------------------------------------------------------------------


\newcommand{\va}{\vct{a}}
\newcommand{\vb}{\vct{b}}
\newcommand{\vc}{\vct{c}}
\newcommand{\vd}{\vct{d}}
\newcommand{\ve}{\vct{e}}
\newcommand{\vf}{\vct{f}}
\newcommand{\vg}{\vct{g}}
\newcommand{\vh}{\vct{h}}
\newcommand{\vi}{\vct{i}}
\newcommand{\vj}{\vct{j}}
\newcommand{\vk}{\vct{k}}
\newcommand{\vl}{\vct{l}}
\newcommand{\vm}{\vct{m}}
\newcommand{\vn}{\vct{n}}
\newcommand{\vo}{\vct{o}}
\newcommand{\vp}{\vct{p}}
\newcommand{\vq}{\vct{q}}
\newcommand{\vr}{\vct{r}}
\newcommand{\vs}{\vct{s}}
\newcommand{\vt}{\vct{t}}
\newcommand{\vu}{\vct{u}}
\newcommand{\vv}{\vct{v}}
\newcommand{\bvv}{\bvct{v}}
\newcommand{\vw}{\vct{w}}
\newcommand{\vx}{\vct{x}}
\newcommand{\vy}{\vct{y}}
\newcommand{\vz}{\vct{z}}
\newcommand{\bvz}{\bvct{z}}
%
\newcommand{\valpha}{\vct{\alpha}}
\newcommand{\bvalpha}{\bvct{\alpha}}
\newcommand{\vbeta}{\vct{\beta}}
\newcommand{\vepsilon}{\vct{\epsilon}}
\newcommand{\vgamma}{\vct{\gamma}}
\newcommand{\vlambda}{\vct{\lambda}}
\newcommand{\vnu}{\vct{\nu}}
\newcommand{\vmu}{\vct{\mu}}
\newcommand{\bvmu}{\bvct{\mu}}
\newcommand{\vphi}{\vct{\phi}}
\newcommand{\vpsi}{\vct{\psi}}
\newcommand{\vtheta}{\vct{\theta}}
\newcommand{\veta}{\vct{\eta}}
\newcommand{\vomega}{\vct{\omega}}
%
\newcommand{\vzero}{\vct{0}}
\newcommand{\vone}{\vct{1}}

%------------------------------------------------

\newcommand{\mA}{\mtx{A}}
\newcommand{\mB}{\mtx{B}}
\newcommand{\mC}{\mtx{C}}
\newcommand{\mD}{\mtx{D}}
\newcommand{\bmD}{\bmtx{D}}
\newcommand{\mE}{\mtx{E}}
\newcommand{\mF}{\mtx{F}}
\newcommand{\mG}{\mtx{G}}
\newcommand{\mH}{\mtx{H}}
\newcommand{\mJ}{\mtx{J}}
\newcommand{\mK}{\mtx{K}}
\newcommand{\mL}{\mtx{L}}
\newcommand{\mM}{\mtx{M}}
\newcommand{\mN}{\mtx{N}}
\newcommand{\mO}{\mtx{O}}
\newcommand{\mP}{\mtx{P}}
\newcommand{\mQ}{\mtx{Q}}
\newcommand{\mR}{\mtx{R}}
\newcommand{\mS}{\mtx{S}}
\newcommand{\mT}{\mtx{T}}
\newcommand{\mU}{\mtx{U}}
\newcommand{\mV}{\mtx{V}}
\newcommand{\mW}{\mtx{W}}
\newcommand{\mX}{\mtx{X}}
\newcommand{\mY}{\mtx{Y}}
\newcommand{\mZ}{\mtx{Z}}
%
\newcommand{\mDelta}{\mtx{\Delta}}
\newcommand{\mLambda}{\mtx{\Lambda}}
\newcommand{\mPhi}{\mtx{\Phi}}
\newcommand{\mPsi}{\mtx{\Psi}}
\newcommand{\mSigma}{\mtx{\Sigma}}
\newcommand{\mUpsilon}{\mtx{\Upsilon}}
%
\newcommand{\mId}{{\bf I}}
\newcommand{\mEx}{{\bf J}}
\newcommand{\mzero}{{\bf 0}}
\newcommand{\mone}{{\bf 1}}

\newcommand{\mAbar}{\underline{\mtx{A}}}
\newcommand{\mRbar}{\underline{\mtx{R}}}
\newcommand{\vebar}{\underline{\vct{e}}}
\newcommand{\vxbar}{\underline{\vct{x}}}
\newcommand{\vybar}{\underline{\vct{y}}}

%------------------------------------------------

\newcommand{\loF}{\linop{F}}
\newcommand{\loG}{\linop{G}}
\newcommand{\loH}{\linop{H}}

%------------------------------------------------

\newcommand{\setA}{\set{A}}
\newcommand{\setB}{\set{B}}
\newcommand{\setC}{\set{C}}
\newcommand{\setD}{\set{D}}
\newcommand{\setE}{\set{E}}
\newcommand{\setF}{\set{F}}
\newcommand{\setG}{\set{G}}
\newcommand{\setH}{\set{H}}
\newcommand{\setI}{\set{I}}
\newcommand{\setJ}{\set{J}}
\newcommand{\setK}{\set{K}}
\newcommand{\setL}{\set{L}}
\newcommand{\setM}{\set{M}}
\newcommand{\setN}{\set{N}}
\newcommand{\setO}{\set{O}}
\newcommand{\setP}{\set{P}}
\newcommand{\setQ}{\set{Q}}
\newcommand{\setR}{\set{R}}
\newcommand{\setS}{\set{S}}
\newcommand{\setT}{\set{T}}
\newcommand{\setU}{\set{U}}
\newcommand{\setV}{\set{V}}
\newcommand{\setW}{\set{W}}
\newcommand{\setX}{\set{X}}
\newcommand{\setY}{\set{Y}}
\newcommand{\setZ}{\set{Z}}



\begin{document}
\section{Acronyms}
ADMM: Alternating Directions Method of Multipliers

ISTA: Iterative Shrinkage Thresholding Algorithm

FISTA: Fast Iterative Shrinkage Tresholding Algorithm

RGB image: an image with red, blue, and green channels

JPEG: the compression process developed by the Joint Photographic Experts Group

\section{Matrices}
Matrices are noted as bold capital letters: $\mD$, $\mA$, $\mB$, $\mPhi$, $\mS$, $\mSS$, $\mX$, $\mU$, $\mV$.

$\mD$ is the a dictionary. For most of the dissertation, $\mD$ has circulant matrix blocks.

$\mS$ is a collection of signal vectors, either gathering multiple channels or multiple samples.

$\mX$ is a collection of dictionary coefficient vectors, corresponding to multple signal vectors.

$\mA$, $\mB$, $\mU$, and $\mV$ are arbitrary matrices. $\mA$ and $\mB$ are also used as the linear operators in the ADMM contraints.

$\mPhi$ is an arbitrary linear operator (this matrix is also listed under operators, since matrices are linear operators).

$\mT$ a diagonal matrix that has diagonal elements of $1$ for dictionary elements that are not constrained to be zero, and zeros for the other diagonal elements (this matrix is also listed under operators since matrices are linear operators).

$\mW$ converts from RGB to YUV, downsamples the UV channels, and computes the DCT  of $8 \times 8$ blocks. Naturally, this matrix also appears under operators.

$\mSS$ is part of the dictionary for the product dictionary model.

$\mQ$ is used to represent the matrix $\rho\mId + \hat{\mD}^H\hat{\mD}$

$\mXi$ is used to represent the matrix $\rho\mId + \hat{\mD}\hat{\mD}^H$

\section{Vectors}
Vectors are bold and lower-case.

$\vx$, $\vy$ are the primal variables for ADMM.

$\vu$ is the dual variable for ADMM.

$\vc$ is the constraint vector in ADMM.

$\vx$, $\vz$ are the coefficients for dictionary model in ADMM algorithm.

$\vv$ is another primal variable (grouped with $\vz$) used in chapter $4$.

$\vgamma$ is the corresponding dual variable.

$\vx$ and $\vz$ are also used as vectors in the FISTA algorithm.

$\vx$ is also used as an arbitrary vector throughout document.  Context will make clear.

$\vs$ is the signal.

$\vb$ is another arbitrary vector.

$\vu$ and $\vv$ are more arbitrary vectors, usually used in pairs. The may collectively specify a rank-$1$ update to a matrix: $\vu\vv^H$.

$\vf$ is a dictionary filter.

$\vd$ is a column of $\mD$.

$\vq$ is the quality-factor dependent vector used in quantization in JPEG compression.

$\vomega$ is an eigenvector.

$\mR$ is a rescaling matrix, used to scale a normalized dictionary back to its unnormalized form.



\section{Non-Integer Scalars}
Non-integer scalars are usually lower-case script letters that are not bolded. (The exception is the estimate of the Lipshitz constant used to determine stepsize in ISTA and FISTA.) I say "non-integer" not imply that they cannot take on integer values, but merely to differentiate them from the scalars that are required to be nonnegative integers.

$a$ and $b$ are arbitrary scalars.

$\rho$ is a scalar for the ADMM alrogithm that specifies both the weighting of the constraints in the augmented Lagrangian and the stepsize in the dual variable update.

$\alpha$ is the over-relaxation or under-relaxation factor for ADMM

$\lambda$ is the factor for $\operatorname{L}_1$ penalty.  $\lambda$ is also used as the factor for the $\operatorname{L}_2$ penalty on the image gradients for Tikhonov regularization. Context will make clear.

$\mathbb{L}$ is an estimate of the Lipshitz constant, used to determine stepsize in the ISTA and FISTA algorithms.

$\tau$ is the eigenvalue

$r$ and $\omega$ are used to specify momentum stepsize in FISTA and a FISTA-like algorithm. Always appear with superscripts specifying iteration.

$\loss$ the loss (as in the loss function)

\section{Indexing Integers}
$n$ selects the sample

$t$ selects the iteration

$m$ selects the filter

$c$ selects the channel

$\hat{k}$ specifies the frequency

\section{Integer Constants}
$M$ is the number of filters

$C$ is the number of channels

$\hat{K}$ is the number of elements in a single channel of the signal

$\reflectbox{R}$ is a small integer that specifies the rank of the dictionary updates

$L$ is the number of layers

\section{Functions and Operations}
$*$ is used to mean circular convolution, except when discussing boundary handling, which works to make circular convolution and convolution equivalent.

$\cdot^{*}$ is the complex conjugate.

$\cdot^H$ is the Hermitian transpose of a vector or matrix

$\L_{\rho}$ is the augmented Lagrangian function

$\operatorname{S}$ is the shrinkage operator

$\F$ applies the Fourier tranform to each channel and/or filter

$f$ and $g$ are arbitrary convex functions. $f$ may also be used to specify the objective function of a minimization problem. $f$ and $g$ are also used for arbitrary functions that are part of a composite function. Context should make clear.

$\operatorname{q}(\cdot)$ quantizes a vector.

$\indicator_{\text{condition}}$ takes on a value of $0$ when the condition is true and $\infty$ when the condition is false.

$\arg \min$ is the argument minimum of a function 

$\nabla_{a} b$ is the gradient of $b$ in respect to $a$.

$L_1$ is the $L_1$ norm

$L_2$ is the $L_2$ norm

$\mPhi$ is an arbitrary linear operator (this operator is also listed under matrices, since matrices are linear operators).

$\mT$ zeros out all dictionary elements that are constrained to be zero. (This operator is also listed under matrices since matrices are linear operators.)

$\mW$ converts from RGB to YUV, downsamples the UV channels, and computes the DCT  of $8 \times 8$ blocks. (This operator also appears under matrices.)

\section{Superscripts}
Conventionally, superscripts are used to indicate exponents.

However, I also use superscripts for other purposes.  To distinguish these superscripts from exponents, I put them in parenthesis $\vx^{(\cdot)}$.

Superscripts are used to indicate which signal sample (or the corresponding dictionary update).

Superscripts are also used to specify the iteration number.

Finally, superscripts are used on gradients to specify a particular gradient term.

\section{Subscripts}
subscript $m$ specifies the filter. If there are multiple layers, $[m]$ will be used instead of subscript $m$.

subscript $n$ specifies the sample.

Subscript $c$ specifies the channel. If there are multiple layers $[c]$ will be used instead of subscript $c$.

subscript $\ell$ specifies the layer.

subscripts $+$ and $-$ are used to specify the eigenvalues or eigenvectors of a $2 \times 2$ matrix corresponding to the plus or minus in the quadratic formula.

The $\rho$ in $\L_{\rho}$ specifies the scalar weight of the $L_2$ norm related to the affine constraints, used in the augmented Lagrangian function.

Subscript $\cdot_{\text{init}}$ is short for initial value.

Subscript $\cdot_{\text{sc}}$ is short for "scaled", and indicates that the variable in the algorithm is a scaled form of the variable.

Subscript $i$ is used for essentially all other indexing.


\end{document}
