\chapter{Introduction}


\section{Dictionaries and Dictionary Learning}
Dictionary models are powerful and versatile tools, useful in denoising \cite{wohlberg2016convolutional}, classification \cite{kong2012dictionary}, anomaly detection \cite{carroll2017outlier}, super-resolution \cite{polatkan2014bayesian}\cite{gu2015convolutional}, inpainting \cite{papyan2017convolutional}, trajectory-basis non-rigid structure from motion \cite{zhu2013convolutional}\cite{chodosh2020use}, and more. Dictionaries are interpretable, and their models are well-suited for incorporating model assumptions using prior knowledge of the data. In some applications, well-performing dictionaries can be learned from very limited data.

The dictionary model decomposes the signal $\vs$ into the dictionary $\mD$ and coefficients $\vx$.
%
\begin{equation}
\vs \approx \mD\vx
\end{equation}
%
When the dictionary is known, solving for $\vx$ using $\mD$ and $\vs$ is a pursuit algorithm. When the dictionary is unknown, learning the dictionary from $\vs$ is called dictionary learning. In this dissertation, I present a novel dictionary learning algorithm specifically for convolutional dictionaries.

\subsection{Convolutional Dictionaries}
\begin{figure}
	\includegraphics[width=\textwidth]{figures/convolutionalDictionary.png}
	\caption{This is a $1$-dimensional dictionary with convolutional structure. The columns of the dictionary are shifted versions of the dictionary filters.}
	\label{Figure: Convolutional Dictionary}
\end{figure}
The convolutional dictionary model is a dictionary model which imposes a convolutional structure on the dictionary. This structure is shown in Figure \ref{Figure: Convolutional Dictionary}. Note that while the example uses $1$-dimensional convolution, it is straightforward to extend to multi-dimensional convolution through vectorization. The convolutional dictionary model equation can be notated as signals convolved with filters:
%
\begin{equation}
\vs \approx \sum_m \vf_m * \vx_m
\end{equation}
%
where $\vf_m$ is the $m^{\text{th}}$ filter of the dictionary, $\vx_m$ is the coefficients for the $m^{\text{th}}$ filter, and $*$ indicates convolution. This structure is useful for signals where shift-invariance is a desirable characteristic and/or when dealing with variable-length data. Convolutional approaches often outperform patch-based methods. For the rest of this dissertation, I will notate $\sum_m \vf_m * \vx_m$ as $\mD\vx$. The matrix $\mD$ consists of $M$ circulant blocks $\mD_{m}$, each block corresponding to a particular filter $\vf_m$ of the $M$ dictionary filters.

\section{Multi-Layer Dictionaries}
A multi-layer dictionary treats the coefficients as the signal of a subsequent dictionary model. That is,
\begin{equation}
\begin{split}
\vs \approx & \mD_1\vx_1
\\
\vx_1 \approx & \mD_2\vx_2
\\
\vdots & 
\\
\vx_{L - 1} \approx & \mD_L\vx_L
\end{split}
\end{equation}
where $L$ is the number of layers. Pursuit for a multi-layer dictionary model seeks to find the coefficients for all layers $\vx_1,\vx_2,\hdots,\vx_L$. Multi-layer dictionary models have been used in applications such as feature extraction for classification \cite{chen2013deep}\cite{zeiler2010deconvolutional}\cite{pu2014bayesian} and trajectory-basis non-rigid structure from motion \cite{chodosh2020use}. Dictionary models can often be easily extended to allow the researcher or practitioner to inject prior knowledge into the model architecture through the use of convex constraints or known linear operator $\mPhi$. For convolutional dictionaries, the projection operator $\mPhi$ need not have convolutional structure.
%
\begin{equation}
\begin{split}
\vs \approx \mPhi\mD\vx
\\
\operatorname{f}_i(\vx) \leq b_i \text{ } \forall i
\end{split}
\end{equation}
%
In the context of multi-layer dictionary models, these types of dictionary model modifications have been shown to be useful for applications such as interpolation of LIDAR measurements using RGB imagery \cite{murdock2018deep}\cite{chodosh2018deep} and compression-artifact removal \cite{chodosh2020use}.

Some researchers have noted that convolutional neural networks resemble multi-layer dictionaries \cite{papyan2017convolutional}: thresholding is a type of pursuit algorithm, and rectified linear unit activations perform the same operation. A convolutional neural network is a composite function of convolutionals, pooling functions, and activation functions. Over the last decade, convolutional neural networks have reached state-of-the-art performance on a wide variety of tasks. In a way, multi-layer dictionary models can be thought of as a blend of dictionary models and convolutional neural networks, since multi-layer dictionary models mimic the layered structure of convolutional neural networks, while maintaining the interpretability of dictionary models.


\section{Organization of Dissertation}
In chapter $2$, I derive a novel dictionary learning method for multi-channel signals. In chapter $3$, I show how that approach can be adapted to multi-layer dictionary models.  Finally, in chapter $4$, I apply the multi-layer dictionary approach to JPEG compression artifact removal. Chapter $5$ features some practical considerations when implementing these algorithms.

% This is a figure
%\begin{figure}
%	\includegraphics[width=\textwidth]{figures/exampleFigure.png}
%	\caption{This is an example Figure.}
%	\label{Figure in Chapter 1}
%\end{figure}


% This is a table

%\begin{table}
%\caption{This is an example Table.}
%\begin{center}
%\begin{tabular}{ccc}
%x & f(x) & g(x) \\
%\hline
%1 & 6 & 4  \\
%2 & 6 & 3  \\
%3 & 6 & 2  \\
%4 & 6 & 2  \\
%\label{Table in Chapter 1}
%\end{tabular}
%\end{center}
%\end{table}
